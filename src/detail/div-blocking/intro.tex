\begin{frame}{数论分块 / 整除分块}
	\label{mulf:intro:divb}

	\only<1->{整除分块主要是用于快速求 \(\sum_{i=1}^n\lfloor\frac{n}{i}\rfloor\) 的算法}

	\only<2>{朴素求法的时间复杂度自然是 \(O(n)\)}

	\only<3->{为了方便叙述, 我们举个例子:

		令\(n=15\), \(f(i)=\lfloor\frac{n}{i}\rfloor\) 我们可列出下表}

	\only<4->{\begin{tabular}{ll}
			\hline
			\(f(i)\) & \(i\)的范围 \\
			\hline
			15       & \([1,1]\)   \\
			7        & \([2,2]\)   \\
			5        & \([3,3]\)   \\
			3        & \([4,5]\)   \\
			2        & \([6,7]\)   \\
			1        & \([8,15]\)  \\
			\hline
		\end{tabular}}

	\only<5->{我们可以发现, \(f\) 像对应的原象是一段区间, 所以我们可以考虑直接求出对应区间长度, 从而就能快速求值}
\end{frame}


\begin{frame}{算法流程}
	\begin{enumerate}
		\item<1-> 令区间左右端点分别为\(l,r\), 答案为\(S\)
		\item<2-> 初始时, \(l\leftarrow1,\ S\leftarrow0\)
		\item<3-> 计算 \(r\)
		\item<4-> \(S\leftarrow S+(r-l+1)\left\lfloor\frac{n}{l}\right\rfloor\)
		\item<5-> \(l\leftarrow r+1\), 若 \(l\leqslant n\) 则回到 3
	\end{enumerate}

	\only<6>{关键就在于如何求 \(r\)}

	\only<7->{令 \(\left\lfloor\frac{n}{l}\right\rfloor=k_l\), 不难发现
		\[
			r=\max_{ik_l\leqslant n}\{i\}
		\]}

	\only<8->{ 显然
		\[
			r=\left\lfloor\frac{n}{k_l}\right\rfloor=\left\lfloor{n\over{\lfloor\frac{n}{l}\rfloor} }\right\rfloor
		\]}

	\only<9->{\textbf{时间复杂度} 由引理 (\ref{mulf:lm:divb2}) 可知是\(O(\sqrt{n})\)}
\end{frame}


\begin{frame}[fragile,allowframebreaks]{实现}
	\includecode[c++]{get_sum.cpp}
\end{frame}
