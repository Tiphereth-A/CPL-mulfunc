\begin{frame}[fragile,allowframebreaks]{Hydro H1060 绝世傻逼题}
	\label{mulf:example:hydroh1060}

	\textbf{题目描述}

	给定 \(D,R,E,A,M\) (\(1\le D,R,E,A,M\le5\times10^7\)) 求:

	\begin{equation}
		\sum_{d=1}^D\sum_{r=1}^R\sum_{e=1}^E\sum_{a=1}^A\sum_{m=1}^M\frac{\begin{matrix}
				\gcd(d,r,e)\gcd(d,r,a)\gcd(d,r,m) \\
				\gcd(d,e,a)\gcd(d,e,m)\gcd(d,a,m) \\
				\gcd(r,e,a)\gcd(r,e,m)\gcd(r,a,m) \\
				\gcd(e,a,m)
			\end{matrix}}{\begin{matrix}
				\gcd(\lcm(d,r),\lcm(d,e),\lcm(d,a),\lcm(d,m),         \\
				\lcm(r,e),\lcm(r,a),\lcm(r,m),\lcm(e,a),\lcm(e,m),    \\
				\lcm(a,m))^3\gcd(\lcm(d,r,e),\lcm(d,r,a),\lcm(d,r,m), \\
				\lcm(d,e,a),\lcm(d,e,m),\lcm(d,a,m),\lcm(r,e,a),      \\
				\lcm(r,e,m),\lcm(r,a,m),\lcm(e,a,m))
			\end{matrix}}
	\end{equation}

	答案对 \(10^9+7\) 取模

	\textbf{输入格式}

	\textbf{本题输入含有多组数据}

	第一行一个正整数 \(T\), 代表数据组数量.

	接下来 \(T\) 行, 每行五个正整数, 分别代表 \(D,R,E,A,M\)

	对于所有数据, \(1\le T\le2000\), \(1\le D,R,E,A,M\le5\times10^7\)

	\textbf{输出格式}

	输出 \(T\) 个正整数, 代表对应数据组答案

	\textbf{样例输入}

	\includecode[common]{hydroh1060.in}

	\textbf{样例输出}

	\includecode[common]{hydroh1060.ans}
\end{frame}


\begin{frame}{题解 (Part 1)}
	化简后的式子为

	\begin{equation}
		\label{mulf:eq:hydroh1060-1}
		\sum_{(d,r,e,a,m)\in S}{\gcd}^6\{d,r,e,a,m\}
	\end{equation}

	其中 \(S=[1,D]\times[1,R]\times[1,E]\times[1,A]\times[1,M]\)
\end{frame}



\begin{frame}{题解 (Part 2)}
	\only<1-4>{设

		\begin{itemize}
			\item<1-> \(n=\min\{D,R,E,A,M\}\)
			\item<2-> \(G=\gcd\{D,R,E,A,M\}\)
			\item<3-> \[
					S(d)=\left[1,\left\lfloor\frac{D}{d}\right\rfloor\right]\times\left[1,\left\lfloor\frac{R}{d}\right\rfloor\right]\times\left[1,\left\lfloor\frac{E}{d}\right\rfloor\right]\times\left[1,\left\lfloor\frac{A}{d}\right\rfloor\right]\times\left[1,\left\lfloor\frac{M}{d}\right\rfloor\right]
				\]
			\item<4-> \[
					F(x)=\left\lfloor\frac{D}{x}\right\rfloor\left\lfloor\frac{R}{x}\right\rfloor\left\lfloor\frac{E}{x}\right\rfloor\left\lfloor\frac{A}{x}\right\rfloor\left\lfloor\frac{M}{x}\right\rfloor
				\]
		\end{itemize}}

	\only<5->{将式 (\ref{mulf:eq:hydroh1060-1}) 继续化简, 得

		\begin{equation}
			\label{mulf:eq:hydroh1060-2}
			\begin{aligned}
				\sum_{(d,r,e,a,m)\in S}{\gcd}^6\{d,r,e,a,m\} & =\sum_{\sigma=1}^n \sigma^6\sum_{(\alpha,\beta,\gamma,\delta,\epsilon)\in S(1)}[\gcd\{\alpha,\beta,\gamma,\delta,\epsilon\}=\sigma]                      \\
				                                             & =\sum_{\sigma=1}^n \sigma^6\sum_{(\alpha,\beta,\gamma,\delta,\epsilon)\in S(\sigma)}[\gcd\{\alpha,\beta,\gamma,\delta,\epsilon\}=1]                      \\
				                                             & =\sum_{\sigma=1}^n \sigma^6\sum_{(\alpha,\beta,\gamma,\delta,\epsilon)\in S(\sigma)}\sum_{\tau\mid \gcd\{\alpha,\beta,\gamma,\delta,\epsilon\}}\mu(\tau) \\
				                                             & =\sum_{\sigma=1}^n \sigma^6\sum_{\tau=1}^{\left\lfloor\frac{G}{\sigma}\right\rfloor} \mu(\tau)                                                           \\
				                                             & \xlongequal{\psi=\sigma\tau} \sum_{\psi=1}^nF(\psi)(\{n^6\}*\mu)(\psi)
			\end{aligned}
		\end{equation}}
\end{frame}

