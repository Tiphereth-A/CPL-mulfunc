\begin{frame}[fragile]{因子数和因子和的积性}
	\only<1->{\begin{theorem}
			\(\tau(n)\) 和 \(\sigma(n)\) 是积性函数
		\end{theorem}}

	\only<2->{\begin{proof}
			令\(m,n\)为两互素正整数, 则

			\[
				\tau(mn)=\sum_{d_1\mid m}\sum_{d_2\mid n}1=\left(\sum_{d_1\mid m}1\right)\left(\sum_{d_2\mid n}1\right)=\tau(m)\tau(n)
			\]

			\[
				\sigma(mn)=\sum_{d_1\mid m}\sum_{d_2\mid n}d_1d_2=\left(\sum_{d_1\mid m}d_1\right)\left(\sum_{d_2\mid n}d_2\right)=\sigma(m)\sigma(n)
			\]
		\end{proof}}
\end{frame}


\begin{frame}[fragile]{因子数和因子和的表达式}
	\begin{corollary}
		\only<1->{设 \(n\) 的标准分解式为

		\[
			n=\prod_{i=1}^rp_i^{\alpha_i}
		\]

		则}

		\only<2->{\begin{equation}
				\tau(n)=\prod_{i=1}^r(\alpha_i+1)
			\end{equation}}

		\only<3->{\begin{equation}
				\sigma(n)=\prod_{i=1}^r\frac{p_i^{\alpha_i+1}-1}{p_i-1}
			\end{equation}}
	\end{corollary}
\end{frame}
